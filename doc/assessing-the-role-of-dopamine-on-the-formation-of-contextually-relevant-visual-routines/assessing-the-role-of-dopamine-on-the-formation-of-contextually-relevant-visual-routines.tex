\documentclass{article}

\usepackage{arxiv}

\usepackage[utf8]{inputenc} % allow utf-8 input
\usepackage[T1]{fontenc}    % use 8-bit T1 fonts
\usepackage{lmodern}        % https://github.com/rstudio/rticles/issues/343
\usepackage{hyperref}       % hyperlinks
\usepackage{url}            % simple URL typesetting
\usepackage{booktabs}       % professional-quality tables
\usepackage{amsfonts}       % blackboard math symbols
\usepackage{nicefrac}       % compact symbols for 1/2, etc.
\usepackage{microtype}      % microtypography
\usepackage{graphicx}

\title{Assessing the influence of dopamine and mindfulness on routines
in visual search}

\author{
    Kelly G. Garner
   \\
    School of Psychology \\
    University of New South Wales \\
  Sydney, NSW \\
  \texttt{\href{mailto:kelly.grace.garner@unsw.edu.au}{\nolinkurl{kelly.grace.garner@unsw.edu.au}}} \\
   \And
    Li-Ann Leow
   \\
    School of Psychology \\
    The University of Queensland \\
  St.~Lucia, QLD \\
  \texttt{} \\
   \And
    Aya Uchida
   \\
    School of Psychology \\
    The University of Queensland \\
  St.~Lucia, QLD \\
  \texttt{} \\
   \And
    Christopher Nolan
   \\
    School of Psychology \\
    The University of New South Wales \\
  Sydney, UNSW \\
  \texttt{} \\
   \And
    Ole Jensen
   \\
    Center for Human Brain Health \\
    University of Birmingham \\
  Birmingham, UK \\
  \texttt{} \\
   \And
    Marta Garrido
   \\
    School of Psychological Sciences \\
    University of Melbourne \\
  Melbourne, VIC \\
  \texttt{} \\
   \And
    Paul E. Dux
   \\
    School of Psychology \\
    The University of Queensland \\
  St.~Lucia, QLD \\
  \texttt{} \\
  }


% tightlist command for lists without linebreak
\providecommand{\tightlist}{%
  \setlength{\itemsep}{0pt}\setlength{\parskip}{0pt}}


% Pandoc citation processing
\newlength{\cslhangindent}
\setlength{\cslhangindent}{1.5em}
\newlength{\csllabelwidth}
\setlength{\csllabelwidth}{3em}
\newlength{\cslentryspacingunit} % times entry-spacing
\setlength{\cslentryspacingunit}{\parskip}
% for Pandoc 2.8 to 2.10.1
\newenvironment{cslreferences}%
  {}%
  {\par}
% For Pandoc 2.11+
\newenvironment{CSLReferences}[2] % #1 hanging-ident, #2 entry spacing
 {% don't indent paragraphs
  \setlength{\parindent}{0pt}
  % turn on hanging indent if param 1 is 1
  \ifodd #1
  \let\oldpar\par
  \def\par{\hangindent=\cslhangindent\oldpar}
  \fi
  % set entry spacing
  \setlength{\parskip}{#2\cslentryspacingunit}
 }%
 {}
\usepackage{calc}
\newcommand{\CSLBlock}[1]{#1\hfill\break}
\newcommand{\CSLLeftMargin}[1]{\parbox[t]{\csllabelwidth}{#1}}
\newcommand{\CSLRightInline}[1]{\parbox[t]{\linewidth - \csllabelwidth}{#1}\break}
\newcommand{\CSLIndent}[1]{\hspace{\cslhangindent}#1}

\begin{document}
\maketitle


\begin{abstract}
Given experience in cluttered but stable visual environments, our
eye-movements adapt to form stereotyped routines that frequent the
locations most likely to offer task-relevant information, while not
confusing routines between similar task-settings. Theoretical insights
suggest that dopamine function and mindfulness may be important
modulators of the formation and deployment of such routines, yet
quantification of their impact remains to be tested in healthy humans.
Over two sessions, participants observed a gaze contingent display
comprised of a 4 x 4 grid of doors, and were instructed to use their
eyes to open doors in order to find hidden targets. Within each session,
doors appeared in either one of two colours, with colour signalling
differing likely target locations (task-settings). We derived measures
for how well participants learned the target locations (accuracy), how
routine was their deployment of eye-movements (stereotypy), and how much
participants confused settings (setting accuracy). Participants received
either Maladopar (dopamine precursor) or placebo (vitamin C) across the
2 sessions, administered under double-blind conditions. Dopamine and
trait mindfulness (assessed by questionnaire) interacted to influence
both accuracy and stereotypy. Increasing dopamine improved accuracy and
reduced stereotypy for individuals scoring high for mindfulness, and
induced the opposite pattern for low mindfulness scorers. Dopamine also
disrupted setting accuracy invariant to mindfulness. The data suggest
that mindfulness modulates the impact of dopamine on the formation of
stereotyped eye-movement routines; low mindfulness may cause settling on
suboptimal routines at the expense of learning. Increasing dopamine
promotes confusion between task-settings, regardless of mindfulness.
These findings provide a link between non-human and human models
regarding the influence of dopamine on the formation of task-relevant
eye-movement routines, and provide novel insights into behaviour-trait
factors that modulate the use of experience when building adaptive
repertoires.
\end{abstract}

\keywords{
    habit
   \and
    sequence
   \and
    dopamine
   \and
    mindfulness
   \and
    eye-movements
  }

\hypertarget{introduction}{%
\section{Introduction}\label{introduction}}

Given stable environmental contingencies, it is adaptive for an organism
to develop routine ways of performing tasks requiring multiple
responses. Dopamine is assumed to play a key role in the neural
computations that underlie the formation of task routines. A large body
of evidence shows that dopaminergic midbrain neurons encode reward
prediction errors (e.g. Schultz, Apicella, and Ljungberg 1993; Hollerman
and Schultz 1998; Waelti, Dickinson, and Schultz 2001), a teaching
signal assumed to compute the value of actions (Sutton and Barto 2018).
A comparable signal in striatum marks the difference between expected
and actual saccadic sequence lengths used by macaques to attain reward
during visual search (Desrochers et al. 2010; Desrochers, Amemori, and
Graybiel 2015). This signal is assumed to reflect a cost-benefit signal
that computes the value of saccadic routines. There also exists a large
body of evidence from rodent and macaque models suggesting that
increased striatal dopamine availability speeds the transition from
goal-directed to habitual control of behaviour (Harmer and Phillips
1998; Nelson and Killcross 2006; Nadel et al. 2021, 2021), the latter of
which is assumed to govern performance of routines (Dezfouli and
Balleine 2012; Dezfouli, Lingawi, and Balleine 2014; Desrochers,
Amemori, and Graybiel 2015; Graybiel and Grafton 2015; Smith and
Graybiel 2016). Although this evidence implicates dopamine in the
formation of task-relevant routines, whether dopamine availability
modulates the formation of saccadic routines in healthy humans remains
an open question.

One way to address this question is to increase dopamine availability
via administration of L-Dopa, a precursor to dopamine. L-Dopa
administration in humans has been associated with elevated positive
prediction errors in striatal blood-oxygenation-level-dependent BOLD
responses (Pessiglione et al. 2006) and with reduction of explorative
choices during instrumental learning (Shohamy et al. 2006; Chakroun et
al. 2020). This suggests that L-Dopa may increase the perceived value of
performed actions by inducing optimistic evaluations of outcomes
(FitzGerald, Dolan, and Friston 2015), possibly by disrupting feedback
processing (Shohamy et al. 2006). Elevating dopamine availability via
L-Dopa may therefore have a comparable impact on the cost-benefit
computations driving the formation of saccadic routines during visual
search. Specifically, L-Dopa may promote an optimistic evaluation of the
performed sequence, increasing the probability that it is adopted as a
routine.

For task-oriented routines to be adaptive, it is also required that they
are not confused between tasks, despite overlap in the situational cues
that mark task environments. Dopamine is assumed to play a modulatory
role in the activation of task-relevant behaviours in response to
relevant situational cues (Budzillo et al. 2017), as well as promoting
the formation of routines. Patients with Parkinson's Disease
consistently show deficits switching between simple sensorimotor tasks
(Cools et al. 2001; Wiecki and Frank 2010), as do healthy participants
who have been administered D2 antagonists (Mehta et al. 2004). Such
findings have been accounted for by assuming that decreased dopamine
causes increased uncertainty about the probability of being in a
specific task-state (Friston et al. 2012). These assumptions are based
on evidence from constrained tasks - i.e.~when single correct responses
are required for given stimuli. In contrast, saccadic routines are often
formed from a self-selected set of many possible eye-movements, and it
is unclear whether dopamine modulates confusion between routines. If an
impact is observed, it is unclear whether the effect is opposite to that
of depleted dopamine, i.e.~does increasing dopamine availability promote
segregation of task routines? Or, does increasing dopamine availability
make it more difficult to switch between routines, thereby creating
confusion between them?

A further but less frequently discussed component of the processes
underlying task routine learning and deployment, is the brain's
representation of task-relevant cues and actions, presumably encoded by
cortex and relayed to striatum for reinforcement (Wickens et al. 2007;
Ashby, Turner, and Horvitz 2010; Bar-Gad et al. 2000; Kirk et al. 2014,
2019). Presumably, the organism that encodes an accurate representation
of cues, actions and outcomes is at an adaptive advantage when forming
and deploying task-relevant routines. A growing body of empirical
evidence suggests that mindfulness may modulate fidelity of such
representations. Theoretically, mindfulness has been defined as a mental
state that emphasises current sensory and internal inputs (Davids 1900;
Shapiro et al. 2006), and as such is well-placed to promote accurate
task-representations. In support of this notion, mindfulness practice
has been associated with increased error monitoring during cognitively
challenging tasks (Andreu et al. 2017), and with greater sensitivity to
dynamics in operant reinforcement contingencies (Chen and Reed 2023;
Reed 2023). This suggests that increased mindfulness corresponds to a
more precise representation of the stimulus and response elements that
make up the task-state.

What could be the modulatory influence of mindfulness on the formation
and deployment of task-relevant routines? Mindfulness appears to have
opposing influences to dopamine on routine learning and task-switching:
individuals low in trait mindfulness are faster to exploit sequential
regularities in stimulus-response tasks (Stillman et al. 2014), and
exploitation of such regularities are assumed to support habitual
responses (Dezfouli and Balleine 2012; Dezfouli, Lingawi, and Balleine
2014). Mindfulness may also promote task-switching; higher levels of
trait mindfulness has been associated with decreased reliance on past
behaviours when stimuli are conserved across tasks that carry different
cognitive demands (Greenberg, Reiner, and Meiran 2012; Kuo and Yeh
2015). Indeed both reinforcement learning (RL) and active inference
frameworks have been used to posit common but opposing mechanistic
actions for both mindfulness and dopamine. In the case of RL,
mindfulness is assumed to attenuate striatal reward prediction errors
(Kirk and Montague 2015; Kirk et al. 2019), possibly via greater
regulation from stronger cortical representations of subjective values
and internal states (Kirk et al. 2014). In the case of active inference,
both dopamine (Friston et al. 2012; FitzGerald, Dolan, and Friston 2015)
and mindfulness (Laukkonen and Slagter 2021; Giommi et al. 2023) are
assumed to modulate the estimate of uncertainty that is used to weight
task-relevant prediction errors. These lines of evidence suggest that
high trait mindfulness may negate the impact of elevated dopamine on
task-relevant routines, yet it remains to be ascertained whether these
assumed links are borne out in a quantitative test.

Thus, using a novel task designed to test the formation and deployment
of task-relevant saccadic routines in humans, we sought to test whether
administration of L-Dopa increased suboptimal routine formation, and
whether increased dopamine positively or negatively modulated switching
between routines. Last we sought to test whether higher levels of trait
mindfulness offered a buffer against the impacts of increased dopamine
availability. To preview the results, L-Dopa decreased accuracy and
promoted routine formation in individuals with low trait-mindfulness,
whereas high trait-mindfulness was associated with the opposite pattern.
Regardless of mindfulness, dopamine hampered switching between routines
by increasing routine confusion.

\hypertarget{methods}{%
\section{Methods}\label{methods}}

\label{sec:Methods}

\hypertarget{participants}{%
\subsection{Participants}\label{participants}}

A total of 40 participants (mean age: 24.5, sd: 5, 30 female, 10 male)
were recruited using the undergraduate and paid SONA pools administered
by the University of Queensland. All procedures were cleared by the
University of Queensland Human Research ethics committee
{[}2017/HE000847{]}, and were conducted in accordance with the National
Statement on Ethical Conduct in Human Research. Participants were over
18 years old, had no known neurological and psychiatric conditions
(assessed by self report), and no contraindications to Levodopa, as
assessed by the Levodopa safety screening questionnaire. Informed
consent was obtained at the start of the first session.

\hypertarget{procedure}{%
\subsection{Procedure}\label{procedure}}

Participants attended two sessions, spaced approximately 1 week apart.
After initial blood pressure (BP) and mood assessments (Bond and Lader
1974), participants received either placebo (vitamin C) or Levodopa
(Madopar 125: 100 mg Levodopa and 25 mg Benserazide Hydrochloride),
crushed and dispersed in orange juice, now referred to as the `placebo'
and `DA' sessions respectively. The solution was prepared by an
experimenter who did not administer the remaining experimental
procedures. This protocol was sufficient to achieve double blinding in
previous work (Chowdhury et al. 2012, 2013). Participants then completed
the Five Facet Mindfulness Questionnaire (Baer et al. 2006) and the
Barratt Impulsivity Scale {[}BIS; Patton, Stanford, and Barratt
(1995){]}, as trait impulsivity scores are associated with midbrain
dopamine D2/D3 receptor availability (Buckholtz et al. 2010). Around 30
minutes after drug administration, participants completed a second BP
and mood rating assessment. Participants then completed the practice
stage of the task, so that the experimental stage began approximately 40
minutes after drug ingestion, within the window of peak plasma
availability {[}Li-Ann, do you have a ref?{]}. At the end of the
session, participants completed the final BP and mood rating assessment
and were asked whether they thought they had been given the active or
placebo drug.

\hypertarget{apparatus}{%
\subsection{Apparatus}\label{apparatus}}

The experimental task was run with custom code\footnote{\url{https://github.com/kel-github/variability-decision-making}},
written using Matlab 2012b (32 bit) and Psychtoolbox v3.0.14, on a
Windows 7 (64-bit) on a Dell Precision T1700 desktop computer, displayed
using a ASUS VG248 monitor. Gaze coordinates (x, y) were sampled at 120
Hz using a monitor-mounted iView Red-m infrared eye tracker
(SensoMotoric Instruments GmbH, Teltow, Germany). Participants were
seated from the monitor at an approximate viewing distance of 57 cm, and
positioned on a chin-rest for the duration of the task.

\hypertarget{experimental-task}{%
\subsection{Experimental Task}\label{experimental-task}}

Each trial began with a fixation dot presented centrally on a grey
screen {[}RGB: 200 200 200{]}. Participants were instructed to fixate on
the dot to begin a trial. After 1000 ms of continuous correct fixation
samples (within 100 pixels of fixation), a square was presented that
comprised 18° of degree visual angle along each length. The square could
be one of four possible colours {[}RGBs: 87, 208, 169; 267, 145, 52;
167, 162, 229; 239, 91, 158{]}. After 1000 ms, a 4 x 4 grid of smaller
squares appeared within the larger square, in a darker version of the
background colour ({[}RGB{]}-50). Each square comprised 2.6° of visual
angle. Participants were instructed that the 4 x 4 grid represented
doors, and that they were to use their eyes to open the doors to find
where the target was hiding. Participants were also instructed that they
were to fixate on a single door to open it. When participants had
fixated on a door for over 300 ms, the door either turned black {[}RGB:
50, 50, 50{]}, to denote the absence of a target, or the target was
displayed and the trial was terminated. If the door had turned black, it
returned to its previous colour as soon as it was detected that the
participant had moved their eyes from the door. Targets were animal
images drawn randomly on each trial from a pool of 100 images taken from
the internet. The time at which the target was available to be found
varied from trial to trial, with the onset being drawn from a uniform
distribution between 500-2000 ms. Once the target was available and the
correct door selected, the target was displayed for 750 ms. Upon
termination of the trial, the grey screen and white fixation cross were
presented (see Fig \ref{fig:taskFig}A).

In each session, participants saw the display in two possible colours.
Participants were instructed that each colour represented a world, and
that the animals had different places they preferred to hide, depending
on the world they were in. There were four possible target locations
within each world, or from here on, each setting. For each setting, 1
door from each quadrant was selected as one of the 4 possible target
locations (see Fig \ref{fig:taskFig}B), with the constraint that target
locations could not overlap between settings. Thus each colour reflected
a setting in which participants could establish a set of task-relevant
eye-movements, i.e.~towards the 4 possible target locations. Note that
within each setting, the target was equally likely to appear behind any
one of the 4 target doors (p=.25) and would never appear behind the
remaining doors (p=0). Colour-target location mappings were
counterbalanced across participants, as was the assignment of colours to
sessions. Participants completed 80 trials in each setting. Eye-movement
calibration and validation was performed every 20 trials. Participants
were also shown the standard QWERTY keyboard and were instructed that
they could press `x' at any time to perform a new calibration and
validation if they felt that their eye-movements were no longer being
registered accurately.

\begin{figure}

{\centering \includegraphics[width=0.7\linewidth]{../../images/DA_ExpTask} 

}

\caption{Experimental Task. A) A single trial where participants use their eyes to open doors to locate a target. B) Contexts and sessions: in each session, participants are exposed to two colour contexts each with 4 unique and equiprobable target locations. Colours and target locations were counterbalanced across participants and sessions. In each session, Levodopa (DA) or placebo is administered under double blind conditions.}\label{fig:taskfig}
\end{figure}

\hypertarget{statistical-approach}{%
\subsection{Statistical Approach}\label{statistical-approach}}

The analysis was designed to assess how well participants learned the
target locations, the extent to which participants formed a routine for
door selections - i.e.~how stereotypical they became in their order of
door-selections, and how well they disambiguated between settings. We
modelled how these elements of performance were modulated by the
dopamine and mindfulness factors. All custom analysis code is available
online\footnote{\url{https://github.com/kel-github/DA_VisRoutes}}. The
analysis was performed using R and RStudio v2022.07.2 (RStudio Team
2020), and can be reproduced in the Neurodesk container environment
(Renton et al. 2022).

\hypertarget{data-cleaning}{%
\subsubsection{Data cleaning}\label{data-cleaning}}

Doors were marked as selected if participants gazed at them for a
duration of at least 300 ms. We assumed that a door could not be
selected twice consecutively, and collapsed any consecutive selections
into a single door selection. Last, as the final door selection of every
trial was fixed (i.e.~finding the target location ends the trial), we
removed the final selection from each trial for the stereotypy (routine)
analysis defined below. We excluded data from one participant whose
total number of door selections was greater than 3 standard deviations
from the mean across both sessions. The remaining 39 datasets were
retained for all of the analyses. Note that this is more inclusive than
our pre-registered plan for data exclusions\footnote{\url{https://osf.io/2y6pk}}.
Based on pilot data, we had planned to exclude participants who scored
\textless{} 65\% accuracy over the course of a session. Analysis of the
final sample suggested that this was too stringent, as this resulted in
the exclusion of 14 of 40 participants.

\hypertarget{accuracy}{%
\subsection{Accuracy}\label{accuracy}}

We first sought to determine the extent to which L-Dopa and mindfulness
influenced the learning of target locations across settings (accuracy).
To compute a measure of accuracy, door selections were classified as
target relevant (TR) for the current setting (i.e.~the setting presented
on trial\_\{t\}, cs), the other setting from that session, not presented
on trial\_\{t\} (os), or neither (n). Note that early in task learning,
an `os' response is a reasonable guess and should not be classified as
incorrect (i.e.~n). The group level mean probabilities of cs, os and n
responses are presented in supplementary Figure X. Data was grouped into
blocks of 10 trials per setting, and grouped across settings, resulting
in 8 blocks of 20 trials. We defined accuracy (acc) as the number of
times participants selected a target door from either setting, relative
to all door selections:

\[
acc = \frac{\sum{(TR_{cs}, TR_{os})}}{\sum{(TR_{cs}, TR_{os}, n)}}
\]

We assessed the influence of block, drug and mindfulness on accuracy
using a Bayesian mixed model approach. Accuracy was assumed to be drawn
from a binomial distribution (1=target door, 0 = non-target door). We
then model the logit function of the probability of drawing a
target-door from the total number of door selections. Note that
resulting regression parameter values reflect changes to the log-odds of
target door selections.

For this and following analyses, we identified the model that best fit
the data, and made inference over the resulting parameters. We report
the 95\% confidence intervals (CIs) of the parameter posteriors, and
assume a reliable effect when the 95\% CIs of the posterior do not
include zero. Models were fit using the BRMS (Bürkner 2017) interface
for Stan (Team, n.d.) and RStan (Stan Development Team 2023). We used
the default weakly informative priors as specified in (Bürkner 2017).
Specifically, fixed and random effect \(\beta\) coefficients were given
a flat prior, intercept and standard deviations were assumed to be drawn
from a student's \(t\) distribution (df=1, location=0, scale=2.5), and
the LKJ-correlation prior with parameter \(\zeta\) \textgreater{} 0 was
used for the parameter covariance matrix. For each model, we checked for
parameter recovery using simulated data. Once fitted, we checked that
the residuals showed no signs of systematic error, that the chains had
converged, and that \(\hat{R}\) values were less than 1.01.

To eschew an overly large model space, and in line with our
pre-registration, we first fit models that contained each possible
combination of the block and drug regressors (and associated random
effects), and found the best model using leave-one-out (LOO) cross
validation (as implemented in Vehtari, Gelman, and Gabry 2017). (Note
that in the pre-registration document we had proposed to compare models
using the deviance information criterion (DIC). As LOO is more robust
than DIC to influential observations, and is readily implemented for use
with BRMS model objects, we opted to use LOO instead of DIC). Upon
identifying the best model, we then added the mindfulness regressor
using all possible combinations, and once again selected the best model
(as evidenced by LOO). Last we controlled for trait impulsivity by
adding BIS scores as a main effect to the winning model. Note that in no
cases did adding BIS scores improve the model. We report the difference
in the expected log posterior density (ELPD) between the next best
models and the winning model, and the ratio of the ELPD difference to
the standard error (SE) of the difference (ELPD:SE), thus a negative
ELPD difference reflects preference for the winning model. The full set
of model comparisons are presented in the supplementary materials.

\hypertarget{setting-accuracy}{%
\subsection{Setting Accuracy}\label{setting-accuracy}}

We next sought to model the impact of L-Dopa and mindfulness on task
confusion. To measure the extent of task confusion, we computed a
measure of setting accuracy (s-acc). This measure indexes how often
participants selected a door that was relevant for the colour setting
displayed on trial t (current setting, cs), relative to how often they
selected a door that was relevant for the setting not displayed on trial
t (i.e.~the other setting from that session, os):

\[
s\mathrm{-}acc = \frac{\sum{TR_{cs}}}{\sum{(TR_{cs}, TR_{os})}}
\]

We modelled the influence of L-Dopa and mindfulness on s-acc using the
Bayesian mixed effects approach described above (Note that in the
pre-registration document we had suggested to include a regressor for
context. Visual inspection of the data showed that s-acc was highly
comparable across contexts {[}see supplemental figures{]}. We therefore
opted to simplify the model space and collapse over this factor).

\hypertarget{stereotypical-door-selections}{%
\subsection{Stereotypical door
selections}\label{stereotypical-door-selections}}

Next, we determined the extent to which door-selections became routine
over the course of the task - specifically, how much the order of door
selections increased in stereotypy, and whether dopamine and mindfulness
modulates the extent of stereotypy. Here we use stereotypy as a proxy
for routine formation, and we define stereotypy as the tendency to
choose doors in the same order, over trials (e.g. Desrochers, Amemori,
and Graybiel 2015).

In order to index stereotypy, we reasoned that stereotypy should result
in an increase in the probability of a subset of door transitions. This
stands in contrast to when making door selections in an entirely
exploratory, or non-stereotyped way, where there should be an even
representation of door transition probabilities. Therefore, the
transition probability matrices of individuals engaged in more
stereotypical door selections should show higher variance than those who
are not engaging in stereotypical door selections. We therefore computed
trial level transition probability matrices, and computed the variance
of each matrix. Variances were then collapsed across settings and trials
to form a stereotypy score for each participant, session and block.

The resulting stereotypy scores were subject to a comparable Bayesian
mixture modelling approach as described above with a few key
differences; the stereotypy scores were assumed to be drawn from a
skewed normal distribution \(\mathcal{N}(\mu, \sigma, \alpha)\) whose
mean (\(\mu\)) was defined by the regression parameters (the
distribution of variance scores can be found in the supplemental
materials). \(\sigma\) was assumed to be drawn from a Student's t
distribution (df=3, location=0, scale=2.5), the skew parameter
(\(\alpha\)) was assumed to be drawn from a normal distribution
\(\mathcal{N}(0,4)\). The remaining priors for the intercept,
beta-coefficients and parameter covariance matrix were defined in the
same manner as for the accuracy data models. As the log-log plot of
variances vs block suggested a power function, analysis was performed on
the logged data. This ensured that the relationship between block and
variance values was best described by a straight line. Identification of
the winning model proceeded as described for the accuracy data above.

\hypertarget{blinding-analyses}{%
\subsection{Blinding analyses}\label{blinding-analyses}}

To determine whether awareness of the dopamine intervention could have
contributed to the findings, the probability of participant ratings were
compared to the expected values assuming chance guessing, using a Chi
Square test. BP and mood ratings were each subject to a session (DA vs
placebo) x timepoint (pre-drug, pre-experiment, post-experiment)
Bayesian repeated measures ANOVA, implemented using the BayesFactor
package for R (Morey, Rouder, and Jamil 2015) using the default priors
(Rouder et al. 2012).

\hypertarget{results}{%
\section{Results}\label{results}}

Overall, mindfulness and DA interacted to show opposing effects on
accuracy and stereotypy, whereas only DA modulated setting accuracy.

\hypertarget{accuracy-1}{%
\subsection{Accuracy}\label{accuracy-1}}

\hypertarget{model-selection}{%
\subsubsection{Model Selection}\label{model-selection}}

First we sought the best model in order to make subsequent inference
over the parameters. The model that best accounted for the experimental
factors contained main fixed effects of block and drug, and random
effects for block x drug. Although this model was only closely preferred
to the next most complex model that contained a block x drug interaction
(ELPD diff = -0.17, ELPD:SE = -0.32), it was strongly preferred to all
other models (min ELPD diff = -674.10, ELPD:SE = -8.35). Adding
mindfulness scores improved the predictive accuracy of the model; the
winning model contained a 3-way block x drug x mindfulness interaction
(and associated two-way interactions and main effect of mindfulness;
ELPD diff = -12.02, ELPD:SE = -1.88). Adding BIS scores did not improve
the predictive value of the model (ELPD diff = -0.16, ELPD:SE = -0.34).
Note that although we draw inferences over parameters from the winning
model, our inferences are the same as if we had used the more complex
model that includes the BIS scores.

\hypertarget{the-effect-of-da-and-mindfulness-on-accuracy}{%
\subsubsection{The effect of DA and mindfulness on
accuracy}\label{the-effect-of-da-and-mindfulness-on-accuracy}}

Having established the best model to account for the data, we next
determine the influence of DA and mindfulness on accuracy by making
inference over the resulting parameters. Accuracy data plotted by block
x drug session (DA va placebo) are shown in Fig \ref{fig:accfig}A.
Critically, the influence of drug on accuracy was impacted by
mindfulness scores. The drug x mindfulness parameter differed reliably
from zero (mean log odds = -0.11, 95\% CI{[}-0.16, -0.06{]}, see Fig
\ref{fig:accfig}E). To better understand this interaction, we computed a
score for each participant that reflected the mean accuracy change due
to the drug session (\(\mu\) acc{[}DA - P{]}). Note that a positive
score indicates that performance was better in the DA session relative
to placebo. Next we examined the relationship between DA-induced
accuracy changes and mindfulness scores. As can be seen in Fig
\ref{fig:accfig}B, there was a positive relationship between DA-induced
accuracy changes and mindfulness; participants scoring higher for
mindfulness showed higher accuracy in the DA relative to the placebo
session, for example, those scoring in the highest quartile showed mean
accuracy scores of 0.67 (95\%CI{[}0.65, 0.70{]}) during the DA session,
relative to mean accuracy scores of 0.62 (95\% CI{[}0.60, 0.63{]})
during the placebo session. Individuals scoring low on mindfulness
showed the opposite pattern (DA mean accuracy = 0.61, 95\% CI{[}0.60,
0.63{]}, placebo mean accuracy = 0.68, 95\%CI{[}0.66, 0.69{]}), note
that Fig \ref{fig:accfig}B shows the difference between these accuracy
scores). Thus the impact of DA on the establishment of task-relevant
eye-movements is dependent on the mindfulness state of the individual.

Participants learned the target door locations over the course of the
sessions, accuracy reliably increased over blocks. Mean accuracy in
block 1 was 0.57 (95\% CI{[}0.55, 0.59{]}), relative to a block 8 mean
of 0.70 (95\% CI{[}0.68, 0.72{]}). The model showed that accuracy
increased by block with an average log odds of = 0.15, (95\% CI{[}0.09,
0.22, Fig \ref{fig:accfig}C). There was also the suggestion of a main
effect of DA (mean log odds = 0.04, 95\% CI{[}-0.001, 0.09, Fig
\ref{fig:accfig}D), however, the impact of DA accuracy is presumably
better explained by the drug x mindfulness interaction.

\begin{figure}

{\centering \includegraphics[width=0.7\linewidth]{../../images/acc_fig} 

}

\caption{The influence of dopamine and mindfulness on accuracy. A) Accuracy (acc) data by block and drug. Circles reflect observed average accuracy, dotted lines reflect the fit of the winning model. B) The association between trait mindfulness (x-axis) and the impact of drug on accuracy [DA-P]. The bottom row shows posterior densities (in log odds) estimated for C) the main effect of block (b), D) the main effect of DA, and E) the drug x mindfulness (m) interaction. DA = dopamine, P = placebo, d = density. Error bars reflect within-subject standard error of the mean [SE].}\label{fig:accfig}
\end{figure}

\hypertarget{setting-accuracy-s-acc}{%
\subsection{Setting accuracy (s-acc)}\label{setting-accuracy-s-acc}}

\hypertarget{model-selection-1}{%
\subsubsection{Model Selection}\label{model-selection-1}}

The best model contained main fixed effects of block and drug, and
random effects for block x drug. Although this model was only closely
preferred to the next most complex model that contained a block x drug
interaction (ELPD diff = -0.66, ELPD:SE = -1.67), it was strongly
preferred to all other models (min ELPD diff = -553.79, ELPD:SE =
-8.35). Adding mindfulness scores improved the predictive accuracy of
the model; the winning model contained an additional main effect of
mindfulness (ELPD diff = -0.13, ELPD:SE = -0.12). Adding BIS scores did
not improve the predictive value of the model (ELPD diff = -0.02,
ELPD:SE = -0.03). Note that although we draw inferences over parameters
from the winning model, our inferences are the same as if we had used
the more complex model that includes the BIS scores.

\hypertarget{drug-and-not-mindfulness-impacts-setting-accuracy}{%
\subsubsection{Drug, and not mindfulness, impacts setting
accuracy}\label{drug-and-not-mindfulness-impacts-setting-accuracy}}

We next determined the influence of DA and mindfulness on setting
accuracy by making inference over the resulting model parameters (see
Fig \ref{fig:caccfig}. DA reduced setting accuracy; accuracy was on
average 0.64 (95\% CI {[}0.63, 0.65{]}) for the DA session, and 0.66
(95\% CI {[}0.65, 0.68{]}) for the placebo session. The log-odds of
selecting a target door that was specific to the current context
increased by a mean log odds of 0.07 (95\% CI{[}0.01, 0.13, Fig
\ref{fig:caccfig}A\&B) for the placebo session, relative to the DA
session. This suggests that DA causes confusion between settings.

Setting accuracy improved over the course of each session; mean accuracy
in block 1 was 0.59 (95\% CI{[}0.58, 0.61{]}), relative to block 8
(mean: 0.69, 95\% CI{[}0.67, 0.71{]}). The model showed that setting
accuracy increased by a mean log odds of 0.13 (95\% CI{[}0.07, 0.19)
over each block (Fig \ref{fig:caccfig}C). In contrast to the overall
accuracy data, mindfulness did show a reliable impact on setting
accuracy (mean log odds = 0.04, 95\% CI{[}-0.04, 0.15{]}).

\begin{figure}

{\centering \includegraphics[width=0.7\linewidth]{../../images/cacc_fig} 

}

\caption{The influence of dopamine and mindfulness on setting accuracy. A) Accuracy (acc) data by block and drug. Circles reflect observed average accuracy, dotted lines show the fit of the winning model. B) Estimated posterior density (in log odds) for the main effect of drug (DA vs P), D) same as in B, but for the main effect of block. DA = dopamine, P = placebo, b = block, d = density. Error bars reflect within-subject standard error of the mean [SE].}\label{fig:caccfig}
\end{figure}

\hypertarget{setting-accuracy-control-analysis}{%
\paragraph{Setting Accuracy Control
Analysis}\label{setting-accuracy-control-analysis}}

DA influences contextual accuracy, which indexes the likelihood of cs
door selections, over os door selections. As we exclude door selections
for locations that are never target relevant (n) from the computation of
setting accuracy, it is important to verify that setting accuracy scores
do indeed reflect confusion between settings (cc + oc), rather than a
general task learning deficit. To address this in an exploratory
analysis, we reasoned that if setting accuracy scores reflected a
general deficit, then `error' door selections (i.e.~non-cs selections)
should be drawn randomly from the remaining doors (os = 4 \& n = 8).
Therefore, when considering only os and n door selections, a general
deficit interpretation suggests that os doors should be selected from
this total set (oc + n) with \(p\) = \(\overline{.333}\), i.e.~at
chance. If setting accuracy scores do tap setting confusion, then os
doors should be selected at a level that is higher than chance. To test
this, we computed for each participant the probability of os selections,
given the set of os and n (\(p_{os}\)), and performed a one-sided
t-test, against a null value of \(p\) = .333. (Note that we opted to use
an NHST approach as we had a point null hypothesis). The \(p_{os}\) data
was unlikely under the null hypothesis (mean = 0.37, 95\% CI{[}0.35,
0.39{]}, \(t\)(38) = 3.62, \(p\) = 0.0004. Therefore, we reject the
hypothesis that the DA induced drop in setting accuracy reflects a
general learning deficit.

\hypertarget{stereotypy-of-door-selections-routine}{%
\subsection{Stereotypy of door selections
(routine)}\label{stereotypy-of-door-selections-routine}}

\hypertarget{model-selection-2}{%
\subsubsection{Model Selection}\label{model-selection-2}}

The model that best accounted for the stereotypy data contained main
fixed effects of block and drug, and random effects for block x drug.
Although this model was only closely preferred to the next most complex
model that contained a block x drug interaction (ELPD diff = -0.26,
ELPD:SE = -1.29), it was strongly preferred to all other models (min
ELPD diff = -130.14, ELPD:SE = -7.71). Adding mindfulness scores
improved the predictive accuracy of the model; the winning model
contained an additional main effect of mindfulness and a drug x
mindfulness interaction (ELPD diff = -3.15, ELPD:SE = -0.92). Adding BIS
scores did not improve the predictive accuracy of the model (ELPD diff =
-0.54, ELPD:SE = -1.41). Note that although we draw inferences over
parameters from the winning model, our inferences are the same as if we
had used the more complex model that includes the BIS scores.

\hypertarget{the-impact-of-drug-and-mindfulness-on-stereotypy}{%
\subsubsection{The impact of drug and mindfulness on
stereotypy}\label{the-impact-of-drug-and-mindfulness-on-stereotypy}}

Mindfulness and dopamine interacted to impact stereotypy; we observed a
reliable drug x mindfulness interaction (mean \(\beta\) = 0.11, 95\%
CI{[}0.02, 0.21, Fig \ref{fig:stereofig}). Mindfulness scores modulated
the stereotypy difference between the DA and placebo sessions; for
example, those scoring in the highest quartile showed mean log variance
scores of -8.74 (95\%CI{[}-8.80, -8.68{]}) during the DA session,
relative to mean log variance scores of -8.68 (95\% CI{[}-8.74,
-8.62{]}) during the placebo session. Individuals scoring low on
mindfulness (lowest quartile) showed the opposite pattern (DA mean
accuracy = -8.69, 95\% CI{[}-8.75, -8.63{]}, placebo mean accuracy =
-8.72, 95\%CI{[}-8.78, -8.66{]}). To visualise this interaction, we
computed a mean variance change score between drug sessions for each
participant (\(\mu\) stereotypy{[}DA - P{]}). Note that a positive score
indicates that performance was more stereotyped in the DA session
relative to placebo. As can be seen in Fig \ref{fig:stereofig}B, there
was a negative relationship between drug-induced stereotypy changes and
mindfulness. Thus the impact of DA on the formation of eye-movement
routines is dependent on the mindfulness state of the individual.

Participants developed routines over the course of the experiment, as
evidenced by a reliable increase in stereotypy over blocks (mean
increase per block: \(\beta\) = 0.32, 95\% CI{[}0.20, 0.43, Fig
\ref{fig:stereofig}C). In line with the interaction of drug x
mindfulness reported above, the main effect of mindfulness suggested a
negative relationship with stereotypy (mindfulness mean \(\beta\) =
-0.15, 95\% CI{[}-0.26, -0.05, Fig \ref{fig:stereofig}D). Overall,
higher mindfulness scores predicted less stereotypy in door-selection
patterns relative to low mindfulness scores.

\begin{figure}

{\centering \includegraphics[width=0.7\linewidth]{../../images/s_fig} 

}

\caption{The influence of dopamine and mindfulness on door selection stereotypy. A) Accuracy (acc) data by block and drug. Circles reflect observed average variance (of the transition matrices), dotted lines show the fit of the winning model. B) The association between trait mindfulness (x-axis) and the impact of drug on variance [DA-P]. The bottom row shows posterior densities (in log odds) estimated for C) the main effect of block (b), D) the main effect of DA, and E) the drug x mindfulness (m) interaction. DA = dopamine, P = placebo, d = density. Error bars reflect within-subject standard error of the mean [SE].}\label{fig:stereofig}
\end{figure}

\hypertarget{on-the-relationship-between-accuracy-and-stereotypy}{%
\paragraph{On the relationship between accuracy and
stereotypy}\label{on-the-relationship-between-accuracy-and-stereotypy}}

Accuracy and stereotypy showed opposing relationships with mindfulness
and DA - higher mindfulness scores were associated with a beneficial
influence of DA on accuracy, and lower levels of stereotypy, relative to
placebo, whereas individuals scoring low on mindfulness showed a
deleterious influence of DA on accuracy, coupled with increased
stereotypy, relative to placebo. As accuracy and stereotypy are
possibly, but not necessarily related, we next sought to ensure that the
observed influences of DA and mindfulness on stereotypy was not being
driven by accuracy, in an exploratory analyses. First we reasoned that
such a pattern of results could be observed if the measures of accuracy
and stereotypy reflected a direct trade off; i.e.~as accuracy goes up,
stereotypy goes down. A correlation analysis ruled out this possibility.
We computed mean accuracy and stereotypy scores for each participant,
collapsing across all experimental factors, and found that accuracy and
stereotypy were positively related (\(r\)(37) = 0.72, \(p\) = 2.45e-07).
Next, to rule out the contribution of accuracy to the stereotypy
results, we added mean accuracy, computed for each block and drug
condition, as a regressor to the winning model. Adding accuracy as a
regressor both clearly improved the predictive accuracy of the model
(ELPD diff = -110.26, ELPD:SE = -6.12), and served to increase certainty
in the interactive influence of mindfulness x drug on stereotypy.
Specifically, the estimated influence of the interaction increased from
\(\beta\) = 0.11 to \(\beta\) = 0.22 (95\% CI{[}0.14, 0.29{]}). Note
that the pattern of remaining results were also consistent between the
two models. Therefore, the data support the notion that mindfulness and
DA interact to differently influence accuracy and stereotypy when
participants perform task-relevant saccadic routines.

\hypertarget{blinding-check}{%
\subsection{Blinding check}\label{blinding-check}}

Next we checked if participants knew whether they had received Levodopa
or placebo across the two sessions. Participants were asked to report at
the end of each session whether they thought they had received Levodopa
or placebo. Participant responses were coded as either correct for both
sessions (cc: observed N = 7), correct for one session and incorrect for
the other (ci: N = 11), or incorrect for both sessions (ii: N = 8). The
probability of the observed guesses was not statistically unlikely given
the null distribution of chance performance (the null hypothesis
specified \(p\)= .25, .5, .25 for cc, ci, ii respectively, \(\chi^2\)(2,
26) = 0.69, \(p\) =0.71). Note that we were unable to include all the
participants in this analysis owing to missing data. Specifically, due
to a miscommunication in the research team, the blinding check questions
contained `Don't know' as a possible response, for which we are unable
to generate a null hypothesis. We therefore only include participants
who made a guess using the Levodopa and placebo options across both
sessions.

\hypertarget{mood-and-blood-pressure}{%
\subsection{Mood and Blood Pressure}\label{mood-and-blood-pressure}}

We also sought to determine whether DA influenced physiological factors
such as mood and BP. For mood, the winning model contained a main effect
of time-point and no other fixed effects. This model was preferred
relative to next best model, which contained an additional main effect
of drug (BF = 3.76, ±2.14\%) and was substantially preferred over the
null random intercept model (BF = 514549 ±1.23\%).

Mean blood pressure was computed using the formula: Mean BVP = diastolic
blood pressure (DBP) + 1/3 {[}systolic blood pressure (SBP) -- DBP{]}.
For mean BVP, the winning model contained main effects of both
time-point and drug. This model was barely preferred to the next best
model which contained a time-point x drug interaction (BF = 1.7
±5.69\%), but was strongly preferred to the random intercept model (BF =
5011975 ±3.76\%). Overall, mean BVP was lower in the Levodopa session
(mean = 2.181.511261, 95\% CI{[}80.3, 82.8{]}), relative to placebo
(mean = 84.5, 95\% CI{[}83.5, 2.185.504042{]}).

\hypertarget{discussion}{%
\section{Discussion}\label{discussion}}

We investigated the impact of L-Dopa administration and trait
mindfulness on the learning of task-relevant behaviour sets, and on the
routine nature of their deployment. Participants opened doors to search
for targets in a gaze-contingent display. The colour of the display
signalled likely target locations, making some locations relevant for
only that colour. We assessed how well participants learned all possible
target locations (accuracy), how routine was the order of door
selections across trials (stereotypy), and how well participants learned
to segregate task-routines (setting-accuracy). Overall, L-Dopa had a
negligible impact on accuracy, but clearly reduced both stereotypy and
setting-accuracy. In the case of accuracy and stereotypy, trait
mindfulness modulated the impact of L-Dopa; high trait mindfulness
corresponded to increased accuracy and decreased stereotypy, for L-Dopa
relative to placebo, whereas low trait mindfulness was associated with
the opposite pattern. These results quantify, for the first time, that
increasing systemic dopamine availability induces a trade-off between
accuracy and stereotypy that is modulated by trait-mindfulness, and that
increased dopamine availability increases routine confusion. These
findings carry implications for our theoretical understanding of how the
brain establishes and switches between task-relevant behavioural
routines, which we outline below.

The current findings offer insight into the relationship between
dopamine and mindfulness. Dopamine and mindfulness have been indirectly
related in both the RL {[}Kirk et al. (2014);
kirkMindfulnessMeditationModulates2015; Kirk et al. (2019){]} and active
inference frameworks (Friston et al. 2012; FitzGerald, Dolan, and
Friston 2015; Laukkonen and Slagter 2021; Giommi et al. 2023), yet there
exists no other study to-date that assesses their joint impact on
behaviour. Here we find that L-Dopa and mindfulness jointly modulate
learning and stereotypy, with L-Dopa yielding conditions of decreased
accuracy and increased stereotypy in low trait mindfulness scorers. We
hypothesise that low mindfulness results in poorer sensory-action
representations which renders the individual more susceptible to error
during credit assignment, which is compounded by over-optimistic
crediting induced by elevated dopamine availability. The result is a
failure to differentiate between the actions that do and do not lead to
reward, and an increased probability of reliance on past behaviours.
This could be manifest via impoverished top-down, cortical regulation of
positive prediction errors in striatum (Kirk et al. 2014), as has been
predicted within an RL framework. The same result could also be
accounted for by a decrease in certainty regarding sensory prediction
errors occurring with low mindfulness (Laukkonen and Slagter 2021;
Giommi et al. 2023), in tandem with dopamine inducing inflated certainty
regarding reward outcomes (FitzGerald, Dolan, and Friston 2015), as has
been suggested via the active inference framework.

Note that the two accounts predict comparable outcomes so we are unable
to differentiate between them with the current data. However, the
current findings do constrain these accounts regarding the extent of
overlap between the actions of dopamine availability and mindfulness.
Increased dopamine availability increased routine confusion, regardless
of trait mindfulness. Therefore, there are limitations to the modulatory
influence of mindfulness on the actions of dopamine. The establishment
and maintenance of a task-set is assumed to reflect a superordinate
representation of a goal and the set of actions required to attain that
goal (Schumacher and Hazeltine 2016; Desrochers et al. 2016; Sutton and
Barto 2018; Vaidya et al. 2021; Lee, Hazeltine, and Jiang 2022). The
current data suggest that while dopamine and trait mindfulness can
jointly modulate the learning and execution of subordinate
representations, i.e.~the set of actions used, mindfulness does not
modulate the impact of dopamine on superordinate task representations,
at least under the current task conditions. Future work should determine
whether these observed limits in the modulatory influence of mindfulness
are due to a limited locus of effect, or are due to increased
vulnerability to the impacts of dopamine at superordinate levels of
representation.

The finding that L-Dopa increased routine confusion suggests that
dopamine modulates switching between tasks requiring multiple responses,
as well as between experimentally constrained sensorimotor tasks as has
been shown previously (Cools et al. 2001; Mehta et al. 2004; Wiecki and
Frank 2010). Collectively, these findings point to a U-shaped function
linking dopamine levels and task-switching impairments, in that depleted
and inflated levels of dopamine result in greater task-switching
deficits. This observation informs theoretical accounts of the
relationship between dopamine and an agent's ability to infer the
current task state, which have previously only considered the impacts of
depleted dopamine (Friston et al. 2012). However, as the currently
studied behaviours are more complex than the constrained sensorimotor
tasks that are typically used in task-switching studies, future work
should verify whether L-Dopa administration comparably impacts
task-switching in simple sensorimotor tasks, and whether depleted
dopamine impacts switching between tasks requiring multiple responses.
This will determine whether the relationship between dopamine and
task-switching is comparable across tasks or is task dependent.

To minimise routine confusion, an agent must maintain a representation
of the actions required to achieve the task goal, and must associate
this representation to the correct task cues. We found that L-Dopa
consistently increased the probability that actions from a non-relevant
task-set would be selected during current task performance, whereas the
probability that an erroneous action was selected varied across
individuals. Therefore, the most consistent locus of task-set confusion
is between actions that have been credited as successful in either
task-context. What remains to be determined is whether L-Dopa caused
task-interference, or whether L-Dopa attenuates the ability to associate
successful actions with the appropriate situational cues. If the latter
is true, then L-Dopa would have caused individuals to learn one task,
that did not incorporate the colour cue as a relevant disambiguating
signal. We seek to arbitrate between these possibilities in future work.

In contrast to expectations, L-Dopa lead to an overall reduction in
stereotypy in door selections, suggesting that increased dopamine
availability reduces the probability of forming a routine when
performing multiple responses. This is in contrast to previous findings
showing that increased dopamine speeds the transition to habit formation
(Harmer and Phillips 1998; Nelson and Killcross 2006; Nadel et al. 2021,
2021). As with task-switching studies, such findings are largely based
on rodent models using tasks comprising one or two stimulus-response
associations. Our findings show that in the case of sets of
task-relevant saccades, increasing dopamine does not necessarily lead to
increased habit formation. Moreover, L-Dopa did not improve accuracy
overall, suggesting that our results cannot be solely attributed to
L-Dopa increasing model-based control (Wunderlich, Smittenaar, and Dolan
2012; Kroemer et al. 2019; Deserno et al. 2021), or adjusting the
balance between exploitation and exploration (Kayser et al. 2015;
Chakroun et al. 2020).

What then is the influence of dopamine on the cost/benefit computations
that drive routine formation? In accordance with previous work with
non-human primates (Desrochers et al. 2010; Desrochers, Amemori, and
Graybiel 2015), the current data do suggest that dopamine is a modulator
of the computations that drive routines in humans. However, the current
data also show that the modulatory influence of dopamine is dependent on
the behaviour-trait state of the individual. Specifically, increased
dopamine appears to drive individuals low in mindfulness towards a
stereotypical solution that is suboptimal in terms of accuracy,
suggesting a poor evaluation of sequence costs relative to benefits. In
contrast, individuals high in trait mindfulness show increased accuracy
but reduced stereotypy, suggesting an appropriate crediting of
successful actions, but also suggesting some volatility in their
execution. While the current data demonstrate the applicability of
dopamine signalling to the computations that underlie the formation of
routines, the data also show further work is required to determine the
internal state variables that determine whether increased dopamine
availability will have a positive or negative impact on performance.

The current work is not without limitations. A difference was found in
mean BVP between the L-Dopa and placebo sessions, suggesting more
general physiological differences between the sessions. However,
participants were not able to detect whether they had received L-Dopa or
placebo above what would be expected by chance. Therefore, the
physiological changes appeared to not be subjectively detectable,
lowering the likelihood that they impacted the results. Note that
although the power of our blinding test was lowered owing to missing
data, the remaining N was comparable to sample sizes from previous
investigations into the impact of dopaminergic pharmacological
intervention on decision-making, that employed comparable blinding tests
(Leow et al. 2023; Pine et al. 2010; Wunderlich, Smittenaar, and Dolan
2012).

Although accuracy and stereotypy theoretically need not be correlated,
we did find a moderate positive correlation between the two measures.
Critically, the modulatory influence of mindfulness and dopamine on
stereotypy was found to be larger after accounting for accuracy.
Furthermore, accuracy and stereotypy were at antithesis to each other
with regard to the demonstrated impacts of mindfulness and L-Dopa.
Nonetheless, further work should be done to confirm the dissociable
impact of dopamine and mindfulness on these two aspects of performance.
We shall seek to achieve this in future studies by controlling task
parameters to maintain accuracy, while examining modulations to
stereotypy.

We sought to determine the modulatory influence of dopamine availability
and trait-mindfulness on the formation and deployment of task-relevant
saccadic routines. We found evidence for theoretical assertions that
dopamine and mindfulness share overlap in their locus of influence, but
also demonstrated boundaries in that overlap. Mindfulness modulated the
impact of dopamine on task-learning and routine development, with low
mindfulness individuals being more likely to demonstrate impaired
learning and increased stereotypy during task performance. Invariant to
trait-mindfulness, L-Dopa increased the likelihood of confusion between
task settings, suggesting that dopamine either hampers the binding of
actions to situational cues, or promotes confusion between task-states.
Collectively, these data suggest that the fidelity of situational
representations interact with reinforcement learning systems to drive
the formation of behavioural routines.

\hypertarget{references}{%
\section*{References}\label{references}}
\addcontentsline{toc}{section}{References}

\hypertarget{refs}{}
\begin{CSLReferences}{1}{0}
\leavevmode\vadjust pre{\hypertarget{ref-andreuBehavioralElectrophysiologicalEvidence2017}{}}%
Andreu, Catherine I., Cristóbal Moënne-Loccoz, Vladimir López, Heleen A.
Slagter, Ingmar H. A. Franken, and Diego Cosmelli. 2017. {``Behavioral
and {Electrophysiological Evidence} of {Enhanced Performance Monitoring}
in {Meditators}.''} \emph{Mindfulness} 8 (6): 1603--14.

\leavevmode\vadjust pre{\hypertarget{ref-ashbyCorticalBasalGanglia2010}{}}%
Ashby, F. Gregory, Benjamin O. Turner, and Jon C. Horvitz. 2010.
{``Cortical and Basal Ganglia Contributions to Habit Learning and
Automaticity.''} \emph{Trends in Cognitive Sciences} 14 (5): 208--15.

\leavevmode\vadjust pre{\hypertarget{ref-baerUsingSelfReportAssessment2006}{}}%
Baer, Ruth A., Gregory T. Smith, Jaclyn Hopkins, Jennifer Krietemeyer,
and Leslie Toney. 2006. {``Using {Self-Report Assessment Methods} to
{Explore Facets} of {Mindfulness}.''} \emph{Assessment} 13 (1): 27--45.

\leavevmode\vadjust pre{\hypertarget{ref-bar-gadReinforcementdrivenDimensionalityReduction2000}{}}%
Bar-Gad, I., G. Havazelet-Heimer, J. A. Goldberg, E. Ruppin, and H.
Bergman. 2000.
{``\href{https://www.ncbi.nlm.nih.gov/pubmed/11248944}{Reinforcement-Driven
Dimensionality Reduction--a Model for Information Processing in the
Basal Ganglia}.''} \emph{Journal of Basic and Clinical Physiology and
Pharmacology} 11 (4): 305--20.

\leavevmode\vadjust pre{\hypertarget{ref-bondUseAnalogueScales1974}{}}%
Bond, Alyson, and Malcolm Lader. 1974. {``The Use of Analogue Scales in
Rating Subjective Feelings.''} \emph{British Journal of Medical
Psychology} 47 (3): 211--18.

\leavevmode\vadjust pre{\hypertarget{ref-buckholtzDopaminergicNetworkDifferences2010}{}}%
Buckholtz, Joshua W., Michael T. Treadway, Ronald L. Cowan, Neil D.
Woodward, Rui Li, M. Sib Ansari, Ronald M. Baldwin, et al. 2010.
{``\href{https://www.ncbi.nlm.nih.gov/pmc/articles/PMC3161413}{Dopaminergic
Network Differences in Human Impulsivity}.''} \emph{Science (New York,
N.Y.)} 329 (5991): 532.

\leavevmode\vadjust pre{\hypertarget{ref-budzilloDopaminergicModulationBasal2017}{}}%
Budzillo, Agata, Alison Duffy, Kimberly E. Miller, Adrienne L. Fairhall,
and David J. Perkel. 2017.
{``\href{https://www.ncbi.nlm.nih.gov/pubmed/28507134}{Dopaminergic
Modulation of Basal Ganglia Output Through Coupled
Excitation\textendash inhibition}.''} \emph{Proceedings of the National
Academy of Sciences} 114 (22): 5713--18.

\leavevmode\vadjust pre{\hypertarget{ref-burknerBrmsPackageBayesian2017}{}}%
Bürkner, Paul-Christian. 2017. {``Brms: {An R} Package for {Bayesian}
Multilevel Models Using {Stan}.''} \emph{Journal of Statistical
Software} 80: 1--28.

\leavevmode\vadjust pre{\hypertarget{ref-chakrounDopaminergicModulationExploration2020}{}}%
Chakroun, Karima, David Mathar, Antonius Wiehler, Florian Ganzer, and
Jan Peters. 2020. {``Dopaminergic Modulation of the
Exploration/Exploitation Trade-Off in Human Decision-Making.''} Edited
by Samuel J Gershman, Michael J Frank, Samuel J Gershman, Bruno B
Averbeck, and John Pearson. \emph{eLife} 9 (June): e51260.

\leavevmode\vadjust pre{\hypertarget{ref-chenEffectBriefMindfulness2023a}{}}%
Chen, Xiaosheng, and Phil Reed. 2023. {``The Effect of Brief Mindfulness
Training on the Micro-Structure of Human Free-Operant Responding:
{Mindfulness} Affects Stimulus-Driven Responding.''} \emph{Journal of
Behavior Therapy and Experimental Psychiatry} 79 (June): 101821.

\leavevmode\vadjust pre{\hypertarget{ref-chowdhuryDopamineModulatesEpisodic2012}{}}%
Chowdhury, Rumana, Marc Guitart-Masip, Nico Bunzeck, Raymond J. Dolan,
and Emrah Düzel. 2012.
{``\href{https://www.ncbi.nlm.nih.gov/pmc/articles/PMC3734374}{Dopamine
Modulates Episodic Memory Persistence in Old Age}.''} \emph{The Journal
of Neuroscience: The Official Journal of the Society for Neuroscience}
32 (41): 14193--204.

\leavevmode\vadjust pre{\hypertarget{ref-chowdhuryDopamineRestoresReward2013}{}}%
Chowdhury, Rumana, Marc Guitart-Masip, Christian Lambert, Peter Dayan,
Quentin Huys, Emrah Düzel, and Raymond J. Dolan. 2013.
{``\href{https://www.ncbi.nlm.nih.gov/pmc/articles/PMC3672991}{Dopamine
Restores Reward Prediction Errors in Old Age}.''} \emph{Nature
Neuroscience} 16 (5): 648--53.

\leavevmode\vadjust pre{\hypertarget{ref-coolsEnhancedImpairedCognitive2001}{}}%
Cools, R., R. A. Barker, B. J. Sahakian, and T. W. Robbins. 2001.
{``\href{https://www.ncbi.nlm.nih.gov/pubmed/11709484}{Enhanced or
Impaired Cognitive Function in {Parkinson}'s Disease as a Function of
Dopaminergic Medication and Task Demands}.''} \emph{Cerebral Cortex (New
York, N.Y.: 1991)} 11 (12): 1136--43.

\leavevmode\vadjust pre{\hypertarget{ref-davids1900buddhist}{}}%
Davids, Thomas William Rhys. 1900. \emph{Buddhist Suttas}. Vol. 11.
{Clarendon Press}.

\leavevmode\vadjust pre{\hypertarget{ref-desernoDopamineEnhancesModelfree2021a}{}}%
Deserno, Lorenz, Rani Moran, Jochen Michely, Ying Lee, Peter Dayan, and
Raymond J Dolan. 2021. {``Dopamine Enhances Model-Free Credit Assignment
Through Boosting of Retrospective Model-Based Inference.''} Edited by
Thorsten Kahnt, Christian Büchel, and Roshan Cools. \emph{eLife} 10
(December): e67778.

\leavevmode\vadjust pre{\hypertarget{ref-desrochersHabitLearningNaive2015}{}}%
Desrochers, Theresa M., Ken-ichi Amemori, and Ann M. Graybiel. 2015.
{``\href{https://www.ncbi.nlm.nih.gov/pubmed/26291166}{Habit {Learning}
by {Naive Macaques Is Marked} by {Response Sharpening} of {Striatal
Neurons Representing} the {Cost} and {Outcome} of {Acquired Action
Sequences}}.''} \emph{Neuron} 87 (4): 853--68.

\leavevmode\vadjust pre{\hypertarget{ref-desrochersMonitoringControlTask2016}{}}%
Desrochers, Theresa M., Diana C. Burk, David Badre, and David L.
Sheinberg. 2016. {``The {Monitoring} and {Control} of {Task Sequences}
in {Human} and {Non-Human Primates}.''} \emph{Frontiers in Systems
Neuroscience} 9.

\leavevmode\vadjust pre{\hypertarget{ref-desrochersOptimalHabitsCan2010}{}}%
Desrochers, Theresa M., Dezhe Z. Jin, Noah D. Goodman, and Ann M.
Graybiel. 2010. {``Optimal Habits Can Develop Spontaneously Through
Sensitivity to Local Cost.''} \emph{Proceedings of the National Academy
of Sciences} 107 (47): 20512--17.

\leavevmode\vadjust pre{\hypertarget{ref-dezfouliHabitsActionSequences2012}{}}%
Dezfouli, Amir, and Bernard W. Balleine. 2012. {``Habits, Action
Sequences and Reinforcement Learning.''} \emph{European Journal of
Neuroscience} 35 (7): 1036--51.

\leavevmode\vadjust pre{\hypertarget{ref-dezfouliHabitsActionSequences2014}{}}%
Dezfouli, Amir, Nura W. Lingawi, and Bernard W. Balleine. 2014.
{``Habits as Action Sequences: Hierarchical Action Control and Changes
in Outcome Value.''} \emph{Philosophical Transactions of the Royal
Society B: Biological Sciences} 369 (1655): 20130482.

\leavevmode\vadjust pre{\hypertarget{ref-fitzgeraldDopamineRewardLearning2015}{}}%
FitzGerald, Thomas H. B., Raymond J. Dolan, and Karl Friston. 2015.
{``Dopamine, Reward Learning, and Active Inference.''} \emph{Frontiers
in Computational Neuroscience} 9.

\leavevmode\vadjust pre{\hypertarget{ref-fristonDopamineAffordanceActive2012}{}}%
Friston, Karl J., Tamara Shiner, Thomas FitzGerald, Joseph M. Galea,
Rick Adams, Harriet Brown, Raymond J. Dolan, Rosalyn Moran, Klaas Enno
Stephan, and Sven Bestmann. 2012. {``Dopamine, {Affordance} and {Active
Inference}.''} \emph{PLOS Computational Biology} 8 (1): e1002327.

\leavevmode\vadjust pre{\hypertarget{ref-giommiFlexibleSelfPsychopathology2023}{}}%
Giommi, Fabio, Prisca R. Bauer, Aviva Berkovich-Ohana, Henk Barendregt,
Kirk Warren Brown, Shaun Gallagher, Ivan Nyklíček, et al. 2023. {``The
({In})flexible Self: {Psychopathology}, Mindfulness, and
Neuroscience.''} \emph{International Journal of Clinical and Health
Psychology} 23 (4): 100381.

\leavevmode\vadjust pre{\hypertarget{ref-graybielStriatumWhereSkills2015}{}}%
Graybiel, Ann M., and Scott T. Grafton. 2015.
{``\href{https://www.ncbi.nlm.nih.gov/pubmed/26238359}{The {Striatum}:
{Where Skills} and {Habits Meet}}.''} \emph{Cold Spring Harbor
Perspectives in Biology} 7 (8): a021691.

\leavevmode\vadjust pre{\hypertarget{ref-greenbergMindTrapMindfulness2012}{}}%
Greenberg, Jonathan, Keren Reiner, and Nachshon Meiran. 2012.
{``{`{Mind} the {Trap}'}: {Mindfulness Practice Reduces Cognitive
Rigidity}.''} \emph{PLOS ONE} 7 (5): e36206.

\leavevmode\vadjust pre{\hypertarget{ref-harmerEnhancedAppetitiveConditioning1998}{}}%
Harmer, C. J., and G. D. Phillips. 1998. {``Enhanced Appetitive
Conditioning Following Repeated Pretreatment with d-Amphetamine.''}
\emph{Behavioural Pharmacology} 9 (4): 299--308.

\leavevmode\vadjust pre{\hypertarget{ref-hollermanDopamineNeuronsReport1998}{}}%
Hollerman, Jeffrey R., and Wolfram Schultz. 1998. {``Dopamine Neurons
Report an Error in the Temporal Prediction of Reward During Learning.''}
\emph{Nature Neuroscience} 1 (4): 304--9.

\leavevmode\vadjust pre{\hypertarget{ref-kayserDopamineLocusControl2015}{}}%
Kayser, Andrew S., Jennifer M. Mitchell, Dawn Weinstein, and Michael J.
Frank. 2015. {``Dopamine, {Locus} of {Control}, and the
{Exploration-Exploitation Tradeoff}.''} \emph{Neuropsychopharmacology}
40 (2): 454--62.

\leavevmode\vadjust pre{\hypertarget{ref-kirkMindfulnessTrainingModulates2014}{}}%
Kirk, Ulrich, Xiaosi Gu, Ann H. Harvey, Peter Fonagy, and P. Read
Montague. 2014.
{``\href{https://www.ncbi.nlm.nih.gov/pmc/articles/PMC4140407}{Mindfulness
Training Modulates Value Signals in Ventromedial Prefrontal Cortex
Through Input from Insular Cortex}.''} \emph{NeuroImage} 100 (October):
254--62.

\leavevmode\vadjust pre{\hypertarget{ref-kirkMindfulnessMeditationModulates2015}{}}%
Kirk, Ulrich, and P. Read Montague. 2015. {``Mindfulness Meditation
Modulates Reward Prediction Errors in a Passive Conditioning Task.''}
\emph{Frontiers in Psychology} 6.

\leavevmode\vadjust pre{\hypertarget{ref-kirkShorttermMindfulnessPractice2019}{}}%
Kirk, Ulrich, Giuseppe Pagnoni, Sébastien Hétu, and Read Montague. 2019.
{``Short-Term Mindfulness Practice Attenuates Reward Prediction Errors
Signals in the Brain.''} \emph{Scientific Reports} 9 (1): 6964.

\leavevmode\vadjust pre{\hypertarget{ref-kroemerLDOPAReducesModelfree2019}{}}%
Kroemer, Nils B., Ying Lee, Shakoor Pooseh, Ben Eppinger, Thomas
Goschke, and Michael N. Smolka. 2019. {``L-{DOPA} Reduces Model-Free
Control of Behavior by Attenuating the Transfer of Value to Action.''}
\emph{NeuroImage} 186 (February): 113--25.

\leavevmode\vadjust pre{\hypertarget{ref-kuoResetTaskSet2015}{}}%
Kuo, Chun-Yu, and Yei-Yu Yeh. 2015. {``Reset a Task Set After Five
Minutes of Mindfulness Practice.''} \emph{Consciousness and Cognition}
35 (September): 98--109.

\leavevmode\vadjust pre{\hypertarget{ref-laukkonenManyOneMeditation2021}{}}%
Laukkonen, Ruben E., and Heleen A. Slagter. 2021. {``From Many to
(n)one: {Meditation} and the Plasticity of the Predictive Mind.''}
\emph{Neuroscience \& Biobehavioral Reviews} 128 (September): 199--217.

\leavevmode\vadjust pre{\hypertarget{ref-leeInterferenceIntegrationHierarchical2022}{}}%
Lee, Woo-Tek, Eliot Hazeltine, and Jiefeng Jiang. 2022.
{``\href{https://www.ncbi.nlm.nih.gov/pubmed/35737531}{Interference and
Integration in Hierarchical Task Learning}.''} \emph{Journal of
Experimental Psychology. General}, June.

\leavevmode\vadjust pre{\hypertarget{ref-leowDopamineIncreasesAccuracy2023a}{}}%
Leow, Li-Ann, Lena Bernheine, Timothy J. Carroll, Paul E. Dux, and
Hannah L. Filmer. 2023. {``Dopamine Increases Accuracy and Lengthens
Deliberation Time in Explicit Motor Skill Learning.''} {bioRxiv}.

\leavevmode\vadjust pre{\hypertarget{ref-mehtaImpairedSetshiftingDissociable2004}{}}%
Mehta, Mitul A., Facundo F. Manes, Gianna Magnolfi, Barbara J. Sahakian,
and Trevor W. Robbins. 2004. {``Impaired Set-Shifting and Dissociable
Effects on Tests of Spatial Working Memory Following the Dopamine {D2}
Receptor Antagonist Sulpiride in Human Volunteers.''}
\emph{Psychopharmacology} 176 (3): 331--42.

\leavevmode\vadjust pre{\hypertarget{ref-moreyBayesFactorComputationBayes2015}{}}%
Morey, R. D., J. N. Rouder, and T. Jamil. 2015. {``{BayesFactor}:
{Computation} of {Bayes Factors} for {Common Designs}.''}

\leavevmode\vadjust pre{\hypertarget{ref-nadelOptogeneticStimulationStriatal2021}{}}%
Nadel, J. A., S. S. Pawelko, J. R. Scott, R. McLaughlin, M. Fox, M.
Ghanem, R. van der Merwe, N. G. Hollon, E. S. Ramsson, and C. D. Howard.
2021. {``Optogenetic Stimulation of Striatal Patches Modifies Habit
Formation and Inhibits Dopamine Release.''} \emph{Scientific Reports} 11
(1): 19847.

\leavevmode\vadjust pre{\hypertarget{ref-nelsonAmphetamineExposureEnhances2006}{}}%
Nelson, Andrew, and Simon Killcross. 2006.
{``\href{https://www.ncbi.nlm.nih.gov/pmc/articles/PMC6674135}{Amphetamine
Exposure Enhances Habit Formation}.''} \emph{The Journal of
Neuroscience: The Official Journal of the Society for Neuroscience} 26
(14): 3805--12.

\leavevmode\vadjust pre{\hypertarget{ref-pattonFactorStructureBarratt1995}{}}%
Patton, J. H., M. S. Stanford, and E. S. Barratt. 1995.
{``\href{https://www.ncbi.nlm.nih.gov/pubmed/8778124}{Factor Structure
of the {Barratt} Impulsiveness Scale}.''} \emph{Journal of Clinical
Psychology} 51 (6): 768--74.

\leavevmode\vadjust pre{\hypertarget{ref-pessiglioneDopaminedependentPredictionErrors2006}{}}%
Pessiglione, Mathias, Ben Seymour, Guillaume Flandin, Raymond J. Dolan,
and Chris D. Frith. 2006. {``Dopamine-Dependent Prediction Errors
Underpin Reward-Seeking Behaviour in Humans.''} \emph{Nature} 442
(7106): 1042--45.

\leavevmode\vadjust pre{\hypertarget{ref-pineDopamineTimeImpulsivity2010}{}}%
Pine, Alex, Tamara Shiner, Ben Seymour, and Raymond J. Dolan. 2010.
{``\href{https://www.ncbi.nlm.nih.gov/pubmed/20592211}{Dopamine, {Time},
and {Impulsivity} in {Humans}}.''} \emph{Journal of Neuroscience} 30
(26): 8888--96.

\leavevmode\vadjust pre{\hypertarget{ref-reedFocusedattentionMindfulnessIncreases2023}{}}%
Reed, Phil. 2023. {``Focused-Attention Mindfulness Increases Sensitivity
to Current Schedules of Reinforcement.''} \emph{Journal of Experimental
Psychology: Animal Learning and Cognition} 49: 127--37.

\leavevmode\vadjust pre{\hypertarget{ref-rentonNeurodeskAccessibleFlexible2022}{}}%
Renton, Angela I., Thanh Thuy Dao, David F. Abbott, Saskia Bollmann,
Megan EJ Campbell, Jeryn Chang, Thomas G. Close, Korbinian Eckstein,
Gary F. Egan, and Stefanie Evas. 2022. {``Neurodesk: {An} Accessible,
Flexible, and Portable Data Analysis Environment for Reproducible
Neuroimaging.''} \emph{bioRxiv}, 2022--12.

\leavevmode\vadjust pre{\hypertarget{ref-rouderDefaultBayesFactors2012}{}}%
Rouder, Jeffrey N., Richard D. Morey, Paul L. Speckman, and Jordan M.
Province. 2012. {``Default {Bayes} Factors for {ANOVA} Designs.''}
\emph{Journal of Mathematical Psychology} 56 (5): 356--74.

\leavevmode\vadjust pre{\hypertarget{ref-rstudiocitation}{}}%
RStudio Team. 2020. \emph{{RStudio}: {Integrated} Development
Environment for r}. Manual. {Boston, MA}: {RStudio, PBC.}

\leavevmode\vadjust pre{\hypertarget{ref-schultzResponsesMonkeyDopamine1993}{}}%
Schultz, W., P. Apicella, and T. Ljungberg. 1993.
{``\href{https://www.ncbi.nlm.nih.gov/pmc/articles/PMC6576600}{Responses
of Monkey Dopamine Neurons to Reward and Conditioned Stimuli During
Successive Steps of Learning a Delayed Response Task}.''} \emph{The
Journal of Neuroscience: The Official Journal of the Society for
Neuroscience} 13 (3): 900--913.

\leavevmode\vadjust pre{\hypertarget{ref-schumacherHierarchicalTaskRepresentation2016}{}}%
Schumacher, Eric H., and Eliot Hazeltine. 2016. {``Hierarchical {Task
Representation}: {Task Files} and {Response Selection}.''} \emph{Current
Directions in Psychological Science} 25 (6): 449--54.

\leavevmode\vadjust pre{\hypertarget{ref-shapiroMechanismsMindfulness2006}{}}%
Shapiro, Shauna L., Linda E. Carlson, John A. Astin, and Benedict
Freedman. 2006. {``Mechanisms of Mindfulness.''} \emph{Journal of
Clinical Psychology} 62 (3): 373--86.

\leavevmode\vadjust pre{\hypertarget{ref-shohamyLdopaImpairsLearning2006}{}}%
Shohamy, Daphna, Catherine E. Myers, Kindiya D. Geghman, Jacob Sage, and
Mark A. Gluck. 2006. {``L-Dopa Impairs Learning, but Spares
Generalization, in {Parkinson}'s Disease.''} \emph{Neuropsychologia} 44
(5): 774--84.

\leavevmode\vadjust pre{\hypertarget{ref-smithHabitFormation2016}{}}%
Smith, Kyle S., and Ann M. Graybiel. 2016.
{``\href{https://www.ncbi.nlm.nih.gov/pmc/articles/PMC4826769}{Habit
Formation}.''} \emph{Dialogues in Clinical Neuroscience} 18 (1): 33--43.

\leavevmode\vadjust pre{\hypertarget{ref-standevelopmentteamRStanInterfaceStan2023}{}}%
Stan Development Team. 2023. {``{RStan}: The {R} Interface to {Stan}.''}

\leavevmode\vadjust pre{\hypertarget{ref-stillmanDispositionalMindfulnessAssociated2014}{}}%
Stillman, Chelsea M., Halley Feldman, Caroline G. Wambach, James H.
Howard, and Darlene V. Howard. 2014. {``Dispositional Mindfulness Is
Associated with Reduced Implicit Learning.''} \emph{Consciousness and
Cognition} 28 (August): 141--50.

\leavevmode\vadjust pre{\hypertarget{ref-sutton2018reinforcement}{}}%
Sutton, Richard S, and Andrew G Barto. 2018. \emph{Reinforcement
Learning: {An} Introduction}. {MIT press}.

\leavevmode\vadjust pre{\hypertarget{ref-standevelopmentteamStanModelingLanguage}{}}%
Team, Stan Development. n.d. {``Stan {Modeling Language Users Guide} and
{Reference Manual}.''}

\leavevmode\vadjust pre{\hypertarget{ref-vaidyaNeuralRepresentationAbstract2021}{}}%
Vaidya, Avinash R, Henry M Jones, Johanny Castillo, and David Badre.
2021. {``Neural Representation of Abstract Task Structure During
Generalization.''} Edited by Mimi Liljeholm, Richard B Ivry, Charan
Ranganath, and Sebastian Michelmann. \emph{eLife} 10 (March): e63226.

\leavevmode\vadjust pre{\hypertarget{ref-vehtariPracticalBayesianModel2017}{}}%
Vehtari, Aki, Andrew Gelman, and Jonah Gabry. 2017. {``Practical
{Bayesian} Model Evaluation Using Leave-One-Out Cross-Validation and
{WAIC}.''} \emph{Statistics and Computing} 27 (5): 1413--32.

\leavevmode\vadjust pre{\hypertarget{ref-waeltiDopamineResponsesComply2001}{}}%
Waelti, Pascale, Anthony Dickinson, and Wolfram Schultz. 2001.
{``Dopamine Responses Comply with Basic Assumptions of Formal Learning
Theory.''} \emph{Nature} 412 (6842): 43--48.

\leavevmode\vadjust pre{\hypertarget{ref-wickensDopaminergicMechanismsActions2007}{}}%
Wickens, Jeffery R., Jon C. Horvitz, Rui M. Costa, and Simon Killcross.
2007.
{``\href{https://www.ncbi.nlm.nih.gov/pubmed/17670964}{Dopaminergic
{Mechanisms} in {Actions} and {Habits}}.''} \emph{Journal of
Neuroscience} 27 (31): 8181--83.

\leavevmode\vadjust pre{\hypertarget{ref-wieckiNeurocomputationalModelsMotor2010}{}}%
Wiecki, Thomas V., and Michael J. Frank. 2010.
{``\href{https://www.ncbi.nlm.nih.gov/pubmed/20696325}{Neurocomputational
Models of Motor and Cognitive Deficits in {Parkinson}'s Disease}.''}
\emph{Progress in Brain Research} 183: 275--97.

\leavevmode\vadjust pre{\hypertarget{ref-wunderlichDopamineEnhancesModelBased2012a}{}}%
Wunderlich, Klaus, Peter Smittenaar, and Raymond J. Dolan. 2012.
{``\href{https://www.ncbi.nlm.nih.gov/pmc/articles/PMC3417237}{Dopamine
{Enhances Model-Based} over {Model-Free Choice Behavior}}.''}
\emph{Neuron} 75 (3-4): 418--24.

\end{CSLReferences}

\bibliographystyle{unsrt}
\bibliography{DAeyes.bib}


\end{document}
